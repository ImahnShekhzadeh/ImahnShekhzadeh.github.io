% --- LaTeX CV Template - S. Venkatraman ---

% --- Set document class and font size ---

\documentclass[11pt, a4paper]{article}

% --- Package imports ---

\usepackage{hyperref, enumitem, longtable, amsmath, array, csquotes, graphicx}

% --- Page layout settings ---

% NEW (16-04-21, 19p49)
\usepackage[margin = 0.6in]{geometry}
\setcounter{page}{1}

% Set line spacing
\renewcommand{\baselinestretch}{1.2}

% --- Page formatting settings ---

% Set link colors
\usepackage[dvipsnames]{xcolor}
\hypersetup{colorlinks=true, linkcolor=MidnightBlue, urlcolor=MidnightBlue}

% Set font to Libertine, including math support
% \usepackage{libertine}
% \usepackage[libertine]{newtxmath}

% Define font size and color for section headings
\newcommand{\headingfont}{\Large\color{OliveGreen}}

% --- CV section settings ---

% Note: each section of this table (Education, Awards, Publications etc.) is 
% stored in a two-column table. The left-hand column is narrow (1 inch) and is 
% meant to store dates. The right-hand column is wide (5.2 inches) and stores 
% the main text.  Sections in which each entry might have multiple lines 
% (e.g., Education) are stored in a 'SectionTable' environment). Sections in 
% which each entry might just have one line are stored in a 'SectionTableSingleSpace'
% environment. The only difference between the two environments is the line 
% spacing between each entry. Both environments take one argument, which is the
% title of the section. See main document for how these environments are used.

% Define settings for left-hand column in which dates are printed
\newcolumntype{R}{>{\raggedleft}p{1in}}

% Define 'SectionTable' environment
\newenvironment{SectionTable}[1]{
	\renewcommand*{\arraystretch}{1.7}
	\setlength{\tabcolsep}{10pt}
	\begin{longtable}{Rp{5.2in}} & #1 \\}
	{\end{longtable}\vspace{-.3cm}}

% Define 'SectionTableSingleSpace' environment
\newenvironment{SectionTableSingleSpace}[1]{
	\renewcommand*{\arraystretch}{1.2}
	\setlength{\tabcolsep}{10pt}
	\begin{longtable}{Rp{5.2in}} & #1 \\[0.6em]}
	{\end{longtable}\vspace{-.3cm}}

% --- Document starts here ---

\begin{document}
	
	% --- Name and contact information ---
	
	\hspace{0.5cm}
	\begin{minipage}{0.1\textwidth}
		\vspace{0.6cm}
		\includegraphics[width=\textwidth]{cv_photo.pdf}
	\end{minipage}
	\hspace{-2.9cm}
	\begin{minipage}{0.9\textwidth}
		\begin{SectionTable}{\Huge Imahn Shekhzadeh} & 
			\href{mailto:imahn.shekhzadeh@unige.ch}{imahn.shekhzadeh@unige.ch} $\;\boldsymbol{\cdot}\;$ \href{https://imahnshekhzadeh.github.io}{https://imahnshekhzadeh.github.io}
		\end{SectionTable}
	\end{minipage}
	
	% --- Section: Research interests ---
	
	% \begin{SectionTable}{\headingfont Research Interests}
	%	& Deep generative modeling, computer vision, graph \mbox{neural} networks
	% \end{SectionTable}
	
	% --- Section: Education ---
	
	\begin{SectionTable}{\headingfont Education} 
		
		Oct 2022 -- Present & \textbf{Graduate Researcher \& PhD Candidate in Computer Science}, University of Geneva. \\
		
		Oct 2020 -- Sep 2022 & 
		\textbf{M.Sc.~Physics}, University of Hamburg. GPA:~1.13.\\
		
		2017 -- 2021 & 
		\textbf{B.Sc.~Physics}, University of Hamburg. GPA:~1.49. \\ 
		
		2009 -- 2017 & 
		\textbf{A-studies}, Margaretha-Rothe-Gymnasium, Hamburg. GPA:~1.00. \\
		% Mentors: Professors E, F. \textit{GPA: X.YZ}. \\
		
		% --- Un-comment the next few lines if you want to include some courses you've taken ---
		
		%& \textbf{Selected coursework}
		%\begin{itemize}[itemsep=0pt, leftmargin=*]
		%\item \textit{Statistics}: Asymptotic statistics, Mathematical statistics, Functional data analysis, High-dimensional statistics, Information theory
		%\item \textit{Mathematics}: Measure theory, Functional analysis, Measure-theoretic probability with martingales
		%\end{itemize}
		
	\end{SectionTable}
	
	% --- Section: Awards, scholarships, etc ---
	
	\begin{SectionTableSingleSpace}{\headingfont Honors and Scholarships}
		2017 -- Present & 
		Member of the \href{https://www.mathges.hamburg/Veranstaltungen.html}{Hamburg Mathematical Society} (\textit{Mathematische Gesellschaft in Hamburg}). \\ 
		
		2017 -- Sep 2022 & 
		Scholarship holder of the \href{https://www.studienstiftung.de/en/}{German Academic Scholarship Foundation} (\textit{\mbox{Studienstiftung} des deutschen Volkes}) 
        for my B.Sc.~\& M.Sc.~studies in Physics. % \newline
		% \textit{Maybe this award needs a short description}. 
	\end{SectionTableSingleSpace}
	
	% --- Section: Publications ---
	\begin{SectionTable}{\headingfont Publications} 
        NeurIPS 2023 & \textit{Calibrating Neural Simulation-Based Inference with Differentiable Coverage Probability}. 
        Maciej Falkiewicz*, Naoya Takeishi*, \textbf{Imahn Shekhzadeh*}, \mbox{Antoine Wehenkel}, Arnaud Delaunoy, Gilles Louppe, 
        Alexandros Kalousis. \\

        Journal of \mbox{Instrumentation} 2023 & \textit{L2LFlows: generating high-fidelity 3D calorimeter images}.
        Sascha Diefenbacher, Engin Eren, Frank Gaede, Gregor Kasieczka, Claudius Krause*, \mbox{\textbf{Imahn Shekhzadeh*}}, David Shih. \\

        NeurIPS 2023 ML$4$Science Workshop & \textit{Advancing Generative Modelling of Calorimeter Showers on Three Frontiers}.
        Erik Buhmann, Sascha Diefenbacher, Engin Eren, Frank Gaede, Gregor Kasieczka, William Korcari, Anatolii Korol, 
        Claudius Krause, Katja Krüger, Peter McKeown, \textbf{Imahn Shekhzadeh}, David Shih. \\[18pt]

        & \small (* equal contribution) \normalsize
        
		%\textbf{Title of your most recent research paper} \newline
		%First author, second author, third author, fourth author. \newline
		%\textit{Journal of something or the other}. \\
	\end{SectionTable}
	
	% --- Section: Teaching experience ---
	\newpage 

	\begin{SectionTable}{\headingfont Work Experience}
		Oct 2022 -- Present & \textbf{Teaching Assistant}, University of Applied Sciences Western Switzerland. \newline Courses:~\textit{Introduction to Machine Learning} (Fall 2022 \& 2023), \textit{Statistics for Machine Learning} (Spring 2023) \\
		
		Spring -- Summer 2018 & 
		\textbf{Light \& Schools, Universität Hamburg} \newline
		(Co-)Supervision of school classes for teaching particular physics or computing applications, such as diffraction of light, app development, etc.\\ 
		
		2013 -- 2017 & \textbf{Margaretha-Rothe-Gymnasium, Hamburg} \newline 
		Tutoring of students in Mathematics, Physics and Latin. 
	\end{SectionTable}

	\begin{SectionTable}{\headingfont Further Projects}
		Nov 2022 -- Present & \textbf{MIGRATE} (A Multidisciplinary and InteGRated Approach for geoThermal Exploration), \textit{collaborators}:~Alexandros Kalousis, Riccardo Lanari, Matteo Lupi, Konstantinos Michailos,
        Juan Luis Porras Loría, Domenico Montanari, Samuele Papeschi, Gurjeet Singh. In an interdisciplinary project, we are studying the automatization of the workflow of ambient noise tomography (ANT) data. 
        This is relevant, since ANT is used for the exploration of geothermal energy, which is a resource potentially available anywhere and at any time. The current ANT workflow, however, heavily relies on 
        simplified assumptions, and the amount of data poses a computational strain, which is where ML methods can help.\\
        
        ML Lecture Project Apr -- Jul 2021 & \textbf{Music Genre Recognition,} \textit{supervised by}:~Prof.~Christina Brandt. In the Master lecture \enquote{Machine Learning}, I worked with two other 
        students on music genre recognition, i.e.~the classification of a music genre from raw audio data. We used both convolutional and recurrent neural networks and preprocessed the audio files into 
        Mel spectograms, which are visual representations of sound. Code:~\url{https://gitlab.com/Imahn/music-genre-recognition}.
		
	\end{SectionTable}
    
    \begin{SectionTable}{\headingfont Skills}
		& \textbf{Programming languages} \newline
		Python, Git \& \LaTeX (proficient), \texttt{C}/\texttt{C++} \& Java (basics) \\
        
        & \textbf{Libraries} \newline
        PyTorch (proficient), TensorFlow (good), Jax (basics) \\
		
		& \textbf{Languages} \newline
		German (native), English \& Farsi/Dari (fluent), French (basics)
	\end{SectionTable}
		
	\vspace{0.6cm}
	% \qquad\qquad\qquad\qquad\ \ Hamburg, 11/29/2021
	Geneva, December 2023
	
	% --- End of CV! ---
	% Original Template: https://www.overleaf.com/latex/templates/academic-cv-template/vqghvksnqdhv
\end{document}